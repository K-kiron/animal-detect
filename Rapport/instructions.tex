\documentclass[11pt]{article}

% This file will be kept up-to-date at the following GitHub repository:
%
% https://github.com/alexhernandezgarcia/mais-latex
%
% Please file any issues/bug reports, etc. you may have at:
%
% https://github.com/alexhernandezgarcia/mais-latex/issues

\usepackage{microtype} % microtypography
\usepackage{booktabs}  % tables
\usepackage{url}  % urls

% AMS math
\usepackage{amsmath}
\usepackage{amsthm}

% With no package options, the submission will be anonymized, the supplemental
% material will be suppressed, and line numbers will be added to the manuscript.
%
% To hide the supplementary material (e.g., for the first submission deadline),
% use the [hidesupplement] option:
%
% \usepackage[hidesupplement]{mais}
%
% To compile a non-anonymized camera-ready version, add the [final] option (for
% the main track) e.g.,
%
% \usepackage[final]{mais}
%
% or
%
% \usepackage[final, hidesupplement]{mais}

\usepackage[final]{mais}

% You may use any reference style as long as you are consistent throughout the
% document. As a default we suggest author--year citations; for bibtex and
% natbib you may use:

\usepackage{natbib}
\bibliographystyle{apalike}

% and for biber and biblatex you may use:

% \usepackage[%
%   backend=biber,
%   style=authoryear-comp,
%   sortcites=true,
%   natbib=true,
%   giveninits=true,
%   maxcitenames=2,
%   doi=false,
%   url=true,
%   isbn=false,
%   dashed=false
% ]{biblatex}
% \addbibresource{references.bib}

\title{Documentation for the \texttt{mais} package}

% The syntax for adding an author is
%
% \author[i]{\nameemail{author name}{author email}}
%
% where i is an affiliation counter. Authors may have
% multiple affiliations; e.g.:
%
% \author[1,2]{\nameemail{Anonymous}{anonymous@example.com}}

\author[1]{\nameemail{Author 1}{email1@example.com}}
\author[2,3]{\nameemail{Author 2}{email2@example.com}}
\author[3]{\nameemail{Author 3}{email3@example.com}}
\author[4]{\nameemail{Author 4}{email4@example.com}}

% if you need to force a linebreak in the author list, prepend an \author entry
% with \\:

\author[3]{\\\nameemail{Author 5}{email5@example.com}}

% Specify corresponding affiliations after authors, referring to counter used in
% \author:

\affil[1]{Institution 1}
\affil[2]{Institution 2}
\affil[3]{Institution 3}
\affil[4]{Institution 4}

% define PDF metadata, aids in accessibility of the resulting PDF
\hypersetup{%
  pdfauthor={MAIS-Conf}, % will be reset to "Anonymous" unless the "final" package option is given
  pdftitle={Documentation for the mais Package},
  pdfsubject={Documentation for mais Package},
  pdfkeywords={MAIS-Conf, LaTeX, style}
}

\begin{document}

\maketitle

The \texttt{mais} package provides a \LaTeX{} style for the extended abstract submissions to the Montreal AI Symposium (MAIS), which will take place on Saturday 17th of September 2022, in Montréal. This document provides some notes regarding the package and tips for typesetting manuscripts. The package and this document is maintained at the following GitHub
repository:

\begin{center}
  \url{https://github.com/alexhernandezgarcia/mais-latex}
\end{center}

Users are encouraged to submit issues, bug reports, etc.\ to:

\begin{center}
  \url{https://github.com/alexhernandezgarcia/mais-latex/issues}
\end{center}

A barebones submission is also available as \texttt{barebones\_submission\_template.tex} in the same repository.

This package has been built upon the \texttt{automl} package, developed and generously open-sourced by Roman Garnett in \url{https://github.com/automl-conf/LatexTemplate}.

\section{Package options}

With no options, the \texttt{mais} package prepares an anonymized manuscript with hidden supplemental material. Two options are supported changing this behavior:

\begin{itemize}
\item \texttt{final} -- produces non-anonymized camera-ready version for distribution and/or publication in the main conference track.
\item \texttt{hidesupplement} -- hides supplementary material (following \verb|\appendix|); for example, for submitting or distributing the main paper without supplement.
\end{itemize}

Note that \texttt{final} may be used in combination with \texttt{hidesupplement} to prepare a non-anonymized version of the main paper with hidden supplement.

\section{Supplemental material}

Please provide supplemental material in the main document. You may begin the supplemental material using \verb|\appendix|. Any content following this command will be suppressed in the final output if the \texttt{hidesupplement} option is given.

\section{Note regarding line numbering at submission time}

To ensure that line numbering works correctly with display math mode, please do \emph{not} use \TeX{} primitives such as \verb|$$| and \texttt{eqnarray}.  (Using these is not good practice anyway.)%
%
\footnote{\url{https://tex.stackexchange.com/questions/196/eqnarray-vs-align}}%
\footnote{\url{https://tex.stackexchange.com/questions/503/why-is-preferable-to}}
%
Please use \LaTeX{} equivalents such as \verb|\[ ... \]| (or \verb|\begin{equation} ... \end{equation}|) and the \texttt{align} environment from the \texttt{amsmath} package.%
%
\footnote{\url{http://tug.ctan.org/info/short-math-guide/short-math-guide.pdf}}

\section{References}

Authors may use any citation style as long as it is consistent throughout the
document. By default, we propose the style defined in `mais.bst` which uses \texttt{natbib/bibtex}, which can be used by including the following at the end of the document:

\begin{verbatim}
\bibliography{references}
\bibliographystyle{mais}
\end{verbatim}

where `mais` refers to the file `mais.bst` and references to `references.bib`, a BibTeX file containing bibliographical entries.

You may create a parenthetical reference with \verb|\citep|, such as appears at the end of this sentence \citep{mitchell2003venus}. You may create a textual reference using \verb|\citet|, as \citet{mitchell2003venus} also demonstrated.

\section{Tables}

We recommend the \texttt{booktabs} package for creating tables, as demonstrated in Table \ref{example_table}. Note that we recommend that tables are centered, that captions appear \emph{above} the tables, and that horizontal rules are placed above and below the table and after the column names. However, horizontal lines for each row are not recommended and vertical rules are discouraged.

\begin{table}
  \caption{An example table using the \texttt{booktabs} package.}
  \label{example_table}
  \centering
  \begin{tabular}{lrr}
    \toprule
    & \multicolumn{2}{c}{Metric} \\
    \cmidrule{2-3}
    Method & Accuracy & Time \\
    \midrule
    Baseline & 10 & 100 \\
    Our method & 100 & 10 \\
    \bottomrule
  \end{tabular}
\end{table}

\section{Figures and subfigures}

The \texttt{mais} style loads the \texttt{subcaption} package, which may be used to create and caption subfigures. Please note that this is \emph{incompatible} with the (obsolete and deprecated) \texttt{subfigure} package. A figure with subfigures is demonstrated in Figure \ref{example_figure}. Note that figure captions appear \emph{below} figures.

Please ensure that all text appearing in figures (axis labels, legends, etc.) is legible and be mindful of the use of colour: favour palettes that are colour-blind friendly and legible if printed in black/white.

\begin{figure}
  \begin{subfigure}[t]{0.5\linewidth}
    \centering
    \framebox{Imagine this is a nice figure}
    \caption{Subfigure caption.}
    \label{example_figure_left}
  \end{subfigure}
  \begin{subfigure}[t]{0.5\linewidth}
    \centering
    \framebox{Imagine this is another nice figure}
    \caption{Another subfigure caption.}
    \label{example_figure_right}
  \end{subfigure}
  \caption{An example figure with subfigures. \subref{example_figure_left}: left figure. \subref{example_figure_right}: right figure.}
  \label{example_figure}
\end{figure}

\section{Pseudocode}
\label{sec:code}

To add pseudocode, you may make use of any package you see fit---the \texttt{mais} package should be compatible with any of them. In particular, you may want to check out the \texttt{algorithm2e}%
%
\footnote{\url{https://ctan.org/pkg/algorithm2e}}
%
and/or the \texttt{algorithmicx}%
%
\footnote{\url{https://ctan.org/pkg/algorithmicx}}
%
packages, both of which can produce nicely typeset pseudocode. You may also wish
to load the \texttt{algorithm}%
%
\footnote{\url{https://ctan.org/pkg/algorithms}}
%
package, which creates an \texttt{algorithm} floating environment you can access
with \verb|\begin{algorithm} ... \end{algorithm}|. This environment supports
\verb|\caption{}|, \verb|\label{}| and \verb|\ref{}|, etc.

\section{Adding acknowledgments}

You may add acknowledgments of funding, etc.\ using the \texttt{acknowledgments} environment. Acknowledgments will be automatically commented out at submission time. An example is given below in the source code for this document; it will be hidden in the \textsc{pdf} unless the \texttt{final} option is given.

\begin{acknowledgements}
  Thank y'all!
\end{acknowledgements}

% print bibliography -- for bibtex / natbib, use:

\bibliography{references}
\bibliographystyle{mais}

% and for biber / biblatex, use:

% \printbibliography

% supplemental material -- everything hereafter will be suppressed during
% submission time if the hidesupplement option is provided!
\appendix

\section{Proof of theorem 1}

This material will be hidden if \texttt{hidesupplement} is provided.

\end{document}
